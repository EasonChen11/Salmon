\documentclass[a4paper,10pt,twocolumn,oneside]{article}
\setlength{\columnsep}{10pt}                                                                    %兩欄模式的間距
\setlength{\columnseprule}{0pt}                                                                %兩欄模式間格線粗細

\usepackage{amsthm}								%定義,例題
\usepackage{amssymb}
%\usepackage[margin=2cm]{geometry}
\usepackage{fontspec}								%設定字體
\usepackage{color}
\usepackage[x11names]{xcolor}
\usepackage{listings}								%顯示code用的
%\usepackage[Glenn]{fncychap}						%排版,頁面模板
\usepackage{fancyhdr}								%設定頁首頁尾
\usepackage{graphicx}								%Graphic
\usepackage{enumerate}
\usepackage{multicol}
\usepackage{titlesec}
\usepackage{amsmath}
\usepackage[CheckSingle, CJKmath]{xeCJK}
\usepackage{savetrees}
\usepackage{xeCJK}   %中文字
\usepackage{array}
\usepackage{xparse}
% \usepackage{CJKulem}

%\usepackage[T1]{fontenc}
\usepackage{amsmath, courier, listings, fancyhdr, graphicx}
\topmargin=0pt
\headsep=5pt
\textheight=780pt
\footskip=0pt
\voffset=-40pt
\textwidth=545pt
\marginparsep=0pt
\marginparwidth=0pt
\marginparpush=0pt
\oddsidemargin=0pt
\evensidemargin=0pt
\hoffset=-42pt

\titlespacing\section{0pt}{-2pt plus 0pt minus 2pt}{-1pt plus 0pt minus 2pt}
\titlespacing\subsection{0pt}{-2pt plus 0pt minus 2pt}{-1pt plus 0pt minus 2pt}
\titlespacing\subsubsection{0pt}{-2pt plus 0pt minus 2pt}{-1pt plus 0pt minus 2pt}


%\renewcommand\listfigurename{圖目錄}
%\renewcommand\listtablename{表目錄} 

%%%%%%%%%%%%%%%%%%%%%%%%%%%%%

\setmainfont{Ubuntu}				%主要字型
\setmonofont{Ubuntu Mono}
\XeTeXlinebreaklocale "zh"						%中文自動換行
\XeTeXlinebreakskip = 0pt plus 1pt				%設定段落之間的距離
\setcounter{secnumdepth}{3}						%目錄顯示第三層

%%%%%%%%%%%%%%%%%%%%%%%%%%%%%
\newcommand\digitstyle{\color{DarkOrchid3}}
\makeatletter
\lst@CCPutMacro\lst@ProcessOther {"2D}{\lst@ttfamily{-{}}{-{}}}
\@empty\z@\@empty

\newtoks\BBQube@token
\newcount\BBQube@length
\def\BBQube@ResetToken{\BBQube@token{}\BBQube@length\z@}
\def\BBQube@Append#1{\advance\BBQube@length\@ne
  \BBQube@token=\expandafter{\the\BBQube@token#1}}

\def\BBQube@ProcessChar#1{%
  \ifnum\lst@mode=\lst@Pmode%
    \ifnum 9<1#1%
      \expandafter\BBQube@Append{\begingroup\digitstyle #1 \endgroup}%
    \else%
      \expandafter\BBQube@Append{#1}%
    \fi%
  \else%
    \expandafter\BBQube@Append{#1}%
  \fi%
}
\def\BBQube@ProcessStringInner#1#2\BBQube@nil{%
  \expandafter\BBQube@ProcessChar{#1}%
  \if\relax\detokenize{#2}\relax%
  \else%
    \expandafter\BBQube@ProcessStringInner#2\BBQube@nil%
  \fi%
}

\def\BBQube@ProcessString#1{\expandafter\BBQube@ProcessStringInner#1\BBQube@nil}

\lst@AddToHook{OutputOther}{%
\BBQube@ResetToken%
\expandafter\BBQube@ProcessString{\the\lst@token}%
\lst@token=\expandafter{\the\BBQube@token}%
}
\makeatother
\lstset{											% Code顯示
language=C++,										% the language of the code
basicstyle=\footnotesize\ttfamily, 						% the size of the fonts that are used for the code
%numbers=left,										% where to put the line-numbers
numberstyle=\footnotesize,						% the size of the fonts that are used for the line-numbers
stepnumber=1,										% the step between two line-numbers. If it's 1, each line  will be numbered
numbersep=5pt,										% how far the line-numbers are from the code
backgroundcolor=\color{white},					% choose the background color. You must add \usepackage{color}
showspaces=false,									% show spaces adding particular underscores
showstringspaces=false,							% underline spaces within strings
showtabs=false,									% show tabs within strings adding particular underscores
frame=false,											% adds a frame around the code
tabsize=2,											% sets default tabsize to 2 spaces
captionpos=b,										% sets the caption-position to bottom
breaklines=true,									% sets automatic line breaking
breakatwhitespace=false,							% sets if automatic breaks should only happen at whitespace
escapeinside={\%*}{*)},							% if you want to add a comment within your code
morekeywords={constexpr},									% if you want to add more keywords to the set
keywordstyle=\bfseries\color{Blue1},
commentstyle=\itshape\color{Red4},
stringstyle=\itshape\color{Green4},
}

%%%%%%%%%%%%%%%%%%%%%%%%%%%%%

\ExplSyntaxOn
\NewDocumentCommand{\captureshell}{som}
 {
  \sdaau_captureshell:Ne \l__sdaau_captureshell_out_tl { #3 }
  \IfBooleanT { #1 }
   {% we may need to stringify the result
    \tl_set:Nx \l__sdaau_captureshell_out_tl
     { \tl_to_str:N \l__sdaau_captureshell_out_tl }
   }
  \IfNoValueTF { #2 }
   {
    \tl_use:N \l__sdaau_captureshell_out_tl
   }
   {
    \tl_set_eq:NN #2 \l__sdaau_captureshell_out_tl
   }
 }

\tl_new:N \l__sdaau_captureshell_out_tl

\cs_new_protected:Nn \sdaau_captureshell:Nn
 {
  \sys_get_shell:nnN { #2 } { } #1
  \tl_trim_spaces:N #1 % remove leading and trailing spaces
 }
\cs_generate_variant:Nn \sdaau_captureshell:Nn { Ne }
\ExplSyntaxOff

\begin{document}
\pagestyle{fancy}
\fancyfoot{}
%\fancyfoot[R]{\includegraphics[width=20pt]{ironwood.jpg}}
\fancyhead[L]{National  Chung  Cheng  University  Salmon}
\fancyhead[R]{\thepage}
\renewcommand{\headrulewidth}{0.4pt}
\renewcommand{\contentsname}{Contents} 
\newcommand{\inputcode}[2]{
    \subsection[#1]{#1 \footnotesize{[\texttt{\captureshell{cpp #2 -dD -P -fpreprocessed | tr -d '[:space:]' | md5sum | cut -c-6}}]}}
    \lstinputlisting{#2}
}

\textbf{
\scriptsize
\begin{multicols}{2}
  \tableofcontents
\end{multicols}
}
%%%%%%%%%%%%%%%%%%%%%%%%%%%%%

%\newpage

\footnotesize
\section{動態規劃}
\subsection{背包問題} 
\lstinputlisting{DP/Bag.cpp}
\subsection{Bitmask DP} 
\lstinputlisting{DP/Bitmask_DP.cpp}
\subsection{硬幣} 
\lstinputlisting{DP/Coin.cpp}
\subsection{編輯距離} 
\lstinputlisting{DP/Edit_Distance.cpp}
\subsection{LCS} 
\lstinputlisting{DP/LCS.cpp}
\subsection{LIS} 
\lstinputlisting{DP/LIS.cpp}
\subsection{Projects} 
\lstinputlisting{DP/Projects.cpp}
\subsection{Removal Game} 
\lstinputlisting{DP/Removal_Game.cpp}

\section{最大流}
\subsection{Dinic} 
\lstinputlisting{Flow/Dinic.cpp}
\subsection{MCMF} 
\lstinputlisting{Flow/MCMF.cpp}

\section{向量}
\subsection{Cross Product} 
\lstinputlisting{Geometry/Cross_Product.cpp}
\subsection{Convex Hull} 
\lstinputlisting{Geometry/Convex_Hull.cpp}

\section{圖論}
\subsection{2-SAT} 
\lstinputlisting{Graph/2-SAT.cpp}
\subsection{DFS 跟 BFS} 
\lstinputlisting{Graph/DFS&BFS.cpp}
\subsection{DSU} 
\lstinputlisting{Graph/DSU.cpp}
\subsection{EulerRoad} 
\lstinputlisting{Graph/EulerRoad.cpp}
\subsection{FloydWarshall} 
\lstinputlisting{Graph/FloydWarshall.cpp}
\subsection{用Bellman找負環} 
\lstinputlisting{Graph/NegCyc_Using_BellmanFord.cpp}
\subsection{最大距離} 
\lstinputlisting{Graph/MaxDistance.cpp}
\subsection{負權最大距離} 
\lstinputlisting{Graph/NegWeights_Max_Distance.cpp}
\subsection{Planet Queries II} 
\lstinputlisting{Graph/Planet_Queries_II.cpp}
\subsection{Planets Cycles} 
\lstinputlisting{Graph/Planets_Cycles.cpp}
\subsection{找環} 
\lstinputlisting{Graph/PosCycle_Finding.cpp}
\subsection{Prim}
\lstinputlisting{Graph/Prim.cpp}
\subsection{SCC 跟拓樸 DP}
\lstinputlisting{Graph/SCC&Topo_DP.cpp}
\subsection{狀態 Dijkstra}
\lstinputlisting{Graph/State_Dijkstra.cpp}
\subsection{Vis Dijkstra}
\lstinputlisting{Graph/Vis_Dijkstra.cpp}

\section{數學}
\subsection{質因數分解} 
\lstinputlisting{Math/Prime.cpp}
\subsection{盧卡斯定理} 
\lstinputlisting{Math/Lucas.cpp}

\section{Queries}
\subsection{BIT} 
\lstinputlisting{Queries/BIT.cpp}
\subsection{Increasing Array Queries}
\lstinputlisting{Queries/Increasing_Array_Queries.cpp}
\subsection{線段樹}
\lstinputlisting{Queries/Segment.cpp}
\subsection{懶標線段樹}
\lstinputlisting{Queries/LazySeg.cpp}
\subsection{莫隊}
\lstinputlisting{Queries/Mo.cpp}
\subsection{Treap}
\lstinputlisting{Queries/Treap.cpp}

\section{搜尋與貪心}
\subsection{二分搜} 
\lstinputlisting{Sorting_and_Searching/Binary_Search.cpp}
\subsection{Concert Ticket}
\lstinputlisting{Sorting_and_Searching/Concert_Ticket.cpp}
\subsection{Restaurant Customers}
\lstinputlisting{Sorting_and_Searching/Restaurant_Customers.cpp}

\section{字串演算法}
\subsection{KMP} 
\lstinputlisting{String/KMP.cpp}

\section{樹論}
\subsection{LCA} 
\lstinputlisting{Tree/LCA.cpp}
\subsection{子樹 DP} 
\lstinputlisting{Tree/Subordinates_DP.cpp}
\subsection{樹重心} 
\lstinputlisting{Tree/Tree_Centroid.cpp}
\subsection{節點距離總和}
\lstinputlisting{Tree/Tree_Sum_Distances.cpp}
\subsection{無權樹直徑} 
\lstinputlisting{Tree/Unweighted_Tree_Distances.cpp}
\subsection{有權樹直徑} 
\lstinputlisting{Tree/Weighted_Tree_Distance.cpp}

\end{document}