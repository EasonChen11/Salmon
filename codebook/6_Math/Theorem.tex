\begin{itemize}
\item 數論分塊可以快速計算一些含有除法向下取整的和式,就是像 $\sum_{i = 1} ^ {n} f(i) g(\left\lfloor\dfrac ni\right\rfloor)$ 的和式。當可以在 $O(1)$ 內計算 $f(r) - f(l)$ 或已經預處理出f的前綴和時,數論分塊就可以在 $O(\sqrt n)$ 的時間內計算上述和式的值。
\item 迪利克雷捲積 $h(x) = \sum_{d \mid x} f(d) g(\frac{x}{d})$
\item 積性函數
    \begin{itemize}
        \item 莫比烏斯函數
            \begin{enumerate}
                \item 
                $$ \begin{aligned}
                    \sum_{d \mid n} \mu (d)
                    &=
                    \begin{cases}
                        1 & \text{for} n = 1\\
                        0 & \text{for} n \neq 0
                    \end{cases}
                \end{aligned} $$

                \item
                $\mu$ 是常數函數 $1$ 的反元素\\
                $\Rightarrow \mu \ast 1 = \epsilon$,$\epsilon (n) \text{只在} n = 1 \text{時為 1,其餘情況皆為 0。}$\\[5pt]
            \end{enumerate}

        \item $\phi$ 歐拉函數: $x$ 以下與 $x$ 互質的數量
        $$ \begin{aligned}
            \phi \ast 1
            &= \sum_{d \mid n} \phi(\dfrac {n}{d}) \ \ \text{質因數分解}\\
            &= \sum_{i = 0} ^ {c} \phi(p ^ i)\\
            &= 1 + p^0(p-1) + p^1(p-1) + ... + p^{c-1}(p-1)\\
            &= p^c\\
            &= id
        \end{aligned} $$
    \end{itemize}

\item 莫比烏斯反演公式
    \begin{itemize}
        \item $f(n)=\sum_{d\mid n}g(d)\Leftrightarrow g(n)=\sum_{d\mid n}\mu(d)f(\frac{n}{d})$
        \item $f(n)=\sum_{n\mid d}g(d)\Leftrightarrow g(n)=\sum_{n\mid d}\mu(\frac{d}{n})f(d)$
    \end{itemize}

\item 例子
$$ \begin{aligned}
    \sum_{i = a}^{b} \sum_{j = c}^{d} [gcd(i, j) = k]\\
    &\Rightarrow \sum_{i = 1}^{x} \sum_{j = 1}^{y} [gcd(i, j) = k]\\
    &= \sum_{i = 1}^{\left\lfloor\dfrac {x}{k} \right\rfloor} \sum_{j = 1}^{\left\lfloor\dfrac {y}{k} \right\rfloor} \epsilon(gcd(i, j))\\
    &= \sum_{i = 1}^{\left\lfloor\dfrac {x}{k} \right\rfloor} \sum_{j = 1}^{\left\lfloor\dfrac {y}{k} \right\rfloor} \sum_{d \mid gcd(i, j)} \mu(d)\\
    &= \sum_{d = 1}^{\infty} \mu(d) \sum_{i = 1}^{\left\lfloor\dfrac {x}{k} \right\rfloor} [d \mid i] \sum {j = 1}{\left\lfloor\dfrac {y}{k} \right\rfloor} [d \mid j] \text{d 可整除 i 時為 1}\\
    &= \sum_{d = 1}^{min(\left\lfloor\dfrac {x}{k} \right\rfloor, \left\lfloor\dfrac {y}{k} \right\rfloor)} \mu(d) \left\lfloor\dfrac {x}{kd} \right\rfloor \left\lfloor\dfrac {y}{kd} \right\rfloor
\end{aligned} $$
\end{itemize}